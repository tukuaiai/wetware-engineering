\documentclass{article}

\usepackage{arxiv}

\usepackage[utf8]{inputenc}
\usepackage[T1]{fontenc}
\usepackage{hyperref}
\usepackage{url}
\usepackage{booktabs}
\usepackage{amsfonts}
\usepackage{nicefrac}
\usepackage{microtype}
\usepackage{cleveref}
\usepackage{graphicx}
\usepackage{natbib}
\usepackage{doi}
\usepackage{listings}
\usepackage{xcolor}
\usepackage{longtable}
\usepackage{array}

% Code listing style
\lstset{
  basicstyle=\ttfamily\small,
  breaklines=true,
  frame=single,
  backgroundcolor=\color{gray!10},
  xleftmargin=2em,
  framexleftmargin=1.5em,
}

\title{Wetware Engineering: Applying Software Engineering Paradigms to Biological System Construction}

\newif\ifuniqueAffiliation
\uniqueAffiliationtrue

\author{
  \href{https://orcid.org/0009-0009-6523-1823}{\includegraphics[scale=0.06]{orcid.pdf}\hspace{1mm}123olp} \\
  \texttt{tukuai.ai@gmail.com}
}

\renewcommand{\shorttitle}{Wetware Engineering}

\hypersetup{
  pdftitle={Wetware Engineering: Applying Software Engineering Paradigms to Biological System Construction},
  pdfsubject={cs.SE, q-bio.OT},
  pdfauthor={123olp},
  pdfkeywords={Software Engineering, Biological Systems, Modular Design, Domain-Specific Language},
}

\begin{document}
\maketitle

\begin{abstract}
Software engineering underwent a paradigm shift from monolithic, handcrafted programs to modular, composable systems over five decades---a transformation enabled by standardized interfaces, package managers, version control, and runtime orchestration. Biological engineering, despite remarkable advances in synthetic biology, organoids, and tissue engineering, remains trapped in an analogous ``pre-modular'' era: each biological system is constructed from scratch, results are difficult to reproduce across laboratories, and there exists no universal language for describing biological component composition.

We propose \textbf{Wetware Engineering}, a cross-disciplinary methodology that systematically transfers software engineering's core abstractions---modularity, interface standardization, dependency management, and runtime orchestration---to biological system construction. This is not merely applying computational tools to biology, but fundamentally reconceptualizing how living systems should be designed, described, and assembled.

Our contribution is threefold: (1) \textbf{Conceptual Framework}: We define the Component-Interface-Runtime triad as the foundational abstraction for modular biological systems, drawing explicit parallels to software architecture patterns. (2) \textbf{Technical Specifications}: We propose Bio-Component Spec, a standardized schema for describing biological modules, and Bio-DSL, a domain-specific language for declarative system composition---both designed following software engineering best practices. (3) \textbf{Paradigm Analysis}: We systematically analyze how software engineering concepts map to biological contexts, identifying both direct translations and fundamental differences requiring novel solutions.

Wetware Engineering represents a paradigm-level contribution: shifting biological system construction from ``artisanal replication'' to ``engineered composition.'' While implementation faces significant biological challenges, establishing this conceptual and methodological foundation is a necessary first step toward reproducible, iterable, and collaborative biological system development.
\end{abstract}

\keywords{Software Engineering \and Biological Systems \and Cross-Disciplinary Methodology \and Modular Design \and Domain-Specific Language \and Systems Biology \and Paradigm Transfer}


\section{Introduction: The Case for Paradigm Transfer}

\subsection{Software Engineering's Modular Revolution}

The history of software engineering is fundamentally a history of rising abstraction levels. In the 1950s, programmers wrote machine code---sequences of binary instructions tied to specific hardware. The introduction of assembly language provided the first abstraction: human-readable mnemonics replacing numeric opcodes. Structured programming in the 1960s abstracted control flow. Object-oriented programming in the 1980s encapsulated data and behavior together. Component-based development in the 1990s enabled binary-level reuse. Service-oriented architecture in the 2000s abstracted deployment locations. Microservices in the 2010s achieved independent deployment and elastic scaling.

Each abstraction level brought transformative benefits:

\begin{table}[h]
\centering
\caption{Evolution of Software Engineering Abstractions}
\begin{tabular}{@{}llll@{}}
\toprule
Era & Abstraction & Key Innovation & Impact \\
\midrule
1950s & Machine code $\rightarrow$ Assembly & Human-readable instructions & 10x productivity \\
1960s & Procedures & Structured programming & Manageable complexity \\
1970s & Modules & Information hiding, interfaces & Team collaboration \\
1980s & Objects & Data + behavior encapsulation & Reusable libraries \\
1990s & Components & Binary reuse (COM, JavaBeans) & Third-party ecosystems \\
2000s & Services & Network-based composition & Enterprise integration \\
2010s & Microservices & Independent deployment & Cloud-native scalability \\
\bottomrule
\end{tabular}
\label{tab:evolution}
\end{table}

The critical insight is that each abstraction level did not merely add convenience---it fundamentally changed what was possible. Before package managers like npm and pip, sharing code meant copying files and manually resolving dependencies. Before containerization, ``it works on my machine'' was an unsolvable problem. Before version control, collaboration meant emailing zip files.

Today, a software developer can declare \texttt{import tensorflow} and instantly access millions of lines of tested, documented, version-controlled code. This is not magic---it is the accumulated result of decades of standardization, tooling, and community building.

\subsection{Biological Engineering's ``Pre-Modular'' State}

Biological engineering in 2025, despite extraordinary advances, remains in a state analogous to software engineering circa 1970. Consider the following comparison:

\begin{table}[h]
\centering
\caption{Software vs. Biological Engineering: Current State}
\begin{tabular}{@{}lll@{}}
\toprule
Software Engineering Concept & Current State in Biology & The Gap \\
\midrule
Standard Library & None & Each lab builds from scratch \\
Package Manager (npm, pip) & None & Cannot declare dependencies \\
Version Control (git) & None & ``This batch differs from last batch'' \\
API Documentation & None & ``Ask the original author'' \\
Unit Testing & None & ``How long will it last?'' \\
CI/CD Pipeline & None & No automated validation \\
Containerization (Docker) & None & Environments not reproducible \\
\bottomrule
\end{tabular}
\label{tab:comparison}
\end{table}

When a tissue engineer wants to combine a muscle actuator with a neural controller, they face challenges that software engineers solved decades ago:

\begin{enumerate}
\item \textbf{No standard interfaces}: The muscle was developed in Lab A with specific culture conditions; the neural tissue in Lab B with different protocols. There is no guarantee they can physically or biochemically connect.

\item \textbf{No dependency declaration}: What exactly does the muscle need? Glucose concentration? Oxygen levels? Stimulation frequency? This information exists in lab notebooks, not machine-readable specifications.

\item \textbf{No version compatibility}: Lab A improved their muscle protocol last month. Does it still work with Lab B's neural tissue? No one knows without re-running experiments.

\item \textbf{No composition language}: How do you describe ``connect muscle output to sensor input, with closed-loop feedback control''? In natural language, buried in a methods section.
\end{enumerate}

The fundamental problem is conceptual: biological systems are treated as \textbf{indivisible wholes} rather than \textbf{composable collections of modules}.

\subsection{Why Paradigm Transfer, Not Just Tool Application}

Existing ``computational biology'' primarily means:
\begin{itemize}
\item Using computers to \textbf{analyze} biological data (bioinformatics)
\item Using algorithms to \textbf{simulate} biological processes (systems biology)
\item Using software to \textbf{control} biological experiments (lab automation)
\end{itemize}

These are valuable but insufficient. They apply software as a tool to biology, without changing how biology itself is engineered.

We propose something fundamentally different:

\begin{quote}
\textbf{Using software engineering's design philosophy to reconceptualize how biological systems are constructed.}
\end{quote}

\begin{table}[h]
\centering
\caption{Levels of Software-Biology Integration}
\begin{tabular}{@{}lll@{}}
\toprule
Level & Existing Approaches & Wetware Engineering \\
\midrule
Tool & Software analyzes biology & --- \\
Method & Algorithms optimize experiments & --- \\
\textbf{Paradigm} & --- & \textbf{Software thinking restructures bioengineering} \\
\bottomrule
\end{tabular}
\label{tab:levels}
\end{table}

The distinction matters. Tools and methods operate within existing paradigms. Paradigm transfer creates new possibilities that were previously inconceivable.

\subsection{Contributions and Paper Structure}

This paper makes the following contributions:

\begin{enumerate}
\item \textbf{Paradigm Definition}: We systematically propose transferring software engineering's core paradigms to biological system construction, articulating why this transfer is both necessary and feasible.

\item \textbf{Abstraction Framework}: We define the Component-Interface-Runtime triad as the foundational abstraction for modular biological systems, with explicit mappings to software architecture patterns.

\item \textbf{Technical Specifications}: We propose Bio-Component Spec v0.1, a standardized schema for describing biological modules, and Bio-DSL, a domain-specific language for declarative system composition.

\item \textbf{Mapping Analysis}: We systematically analyze how software engineering concepts translate to biological contexts, categorizing mappings as Direct, Analogous, or Novel (requiring new solutions).

\item \textbf{Difference Identification}: We identify fundamental differences between software and biological systems that require innovative approaches beyond direct paradigm transfer.
\end{enumerate}

The paper is structured as follows: \S2 defines core abstractions and the Component-Interface-Runtime triad; \S3 presents systematic mappings from software to biological engineering; \S4 details the Bio-Component Specification design; \S5 describes Bio-DSL language design rationale; \S6 analyzes fundamental differences and open challenges; \S7 positions our work relative to existing approaches; \S8 concludes with future directions.


\section{Core Abstractions: The Component-Interface-Runtime Triad}

\subsection{Abstraction as the Essence of Engineering}

Edsger Dijkstra observed: ``The purpose of abstraction is not to be vague, but to create a new semantic level in which one can be absolutely precise.'' This insight captures why abstraction is not merely a convenience but the essence of engineering progress.

Software engineering's success stems from identifying \textbf{correct abstraction boundaries}:
\begin{itemize}
\item Functions abstract instruction sequences
\item Objects abstract data and behavior
\item Interfaces abstract implementation details
\item Services abstract deployment locations
\item Containers abstract operating environments
\end{itemize}

Each abstraction creates a ``semantic level'' where engineers can reason precisely without concerning themselves with lower-level details. A web developer using React does not think about memory allocation; a data scientist using pandas does not think about CPU cache optimization.

The central question for biological engineering is: \textbf{What are the correct abstraction boundaries for living systems?}

We propose that the answer lies in the same triad that revolutionized software: \textbf{Component}, \textbf{Interface}, and \textbf{Runtime}.

\subsection{Component: The Unit of Biological Reuse}

A \textbf{Bio-Component} is a biological unit that:
\begin{itemize}
\item Can exist independently (with appropriate life support)
\item Can receive energy and nutrients (powerable)
\item Can respond to control signals (controllable)
\item Can produce functional outputs (functional)
\item Can report its state (observable)
\end{itemize}

This definition deliberately parallels software component definitions. A software component is similarly self-contained, has defined inputs and outputs, maintains internal state, and can be monitored.

\begin{table}[h]
\centering
\caption{Software to Bio-Component Property Mapping}
\begin{tabular}{@{}ll@{}}
\toprule
Software Component Property & Bio-Component Equivalent \\
\midrule
Encapsulation & Physical boundary, membrane structure \\
Interface & Input/output port definitions \\
State & Physiological state, viability indicators \\
Lifecycle & Culture, activation, maintenance, senescence \\
Dependencies & Nutrients, oxygen, signal inputs \\
Side Effects & Metabolic waste, secretions \\
\bottomrule
\end{tabular}
\label{tab:component-mapping}
\end{table}

Drawing from software architecture patterns, we propose a typology of Bio-Components:

\begin{table}[h]
\centering
\caption{Component Typology}
\begin{tabular}{@{}lll@{}}
\toprule
Type & Software Analogy & Biological Examples \\
\midrule
Actuator & Output device driver & Muscle, gland, ciliated epithelium \\
Sensor & Input device driver & Photoreceptor, mechanoreceptor, chemoreceptor \\
Processor & CPU, logic unit & Ganglion, brain organoid, neural network \\
Storage & Memory, database & Adipose tissue, bone marrow \\
Connector & Network interface & Blood vessel, nerve fiber \\
Metabolic & Power supply & Liver tissue, mitochondria-rich cells \\
\bottomrule
\end{tabular}
\label{tab:typology}
\end{table}

\subsection{Interface: The Contract for Composition}

The Gang of Four's design principle states: ``Program to an interface, not an implementation.'' This principle enabled the explosion of software reuse: as long as components agree on interfaces, their internal implementations can vary independently.

An interface is a \textbf{contract} that defines:
\begin{itemize}
\item What inputs are accepted (preconditions)
\item What outputs are produced (postconditions)
\item What guarantees are maintained (invariants)
\end{itemize}

Biological interfaces are more complex than software interfaces because they operate across multiple physical dimensions simultaneously. We identify four primary dimensions:

\textbf{Power Interface}: How energy and nutrients flow between components
\begin{itemize}
\item Perfusion connections (blood vessel equivalents)
\item Nutrient diffusion surfaces
\item Oxygen delivery mechanisms
\end{itemize}

\textbf{Signal Interface}: How information is exchanged
\begin{itemize}
\item Electrical signals (neural)
\item Chemical signals (hormones, neurotransmitters)
\item Mechanical signals (stretch, pressure)
\item Optical signals (for optogenetic systems)
\end{itemize}

\textbf{Isolation Interface}: How components are protected from each other
\begin{itemize}
\item Immune barriers (preventing rejection)
\item Toxicity isolation (containing harmful metabolites)
\item Electrical isolation (preventing signal crosstalk)
\end{itemize}

\textbf{Mechanical Interface}: How physical forces are transmitted
\begin{itemize}
\item Structural attachments
\item Force transmission surfaces
\item Movement coupling
\end{itemize}

\subsection{Runtime: The Orchestration Layer}

In software systems, the runtime environment handles resource management, lifecycle management, fault handling, monitoring, and coordination. Modern container orchestrators like Kubernetes exemplify sophisticated runtime systems.

A Bio-Runtime must handle analogous responsibilities:

\begin{table}[h]
\centering
\caption{Runtime Responsibilities Mapping}
\begin{tabular}{@{}ll@{}}
\toprule
Software Runtime & Bio-Runtime \\
\midrule
Memory allocation & Nutrient allocation \\
CPU scheduling & Activity timing control \\
Network I/O & Signal routing \\
Health checks & Viability monitoring \\
Auto-restart & Regeneration/replacement triggering \\
Logging & Biomarker time-series recording \\
Load balancing & Workload distribution across redundant modules \\
Fault isolation & Containing necrosis, inflammation \\
\bottomrule
\end{tabular}
\label{tab:runtime}
\end{table}

The perfusion system---delivering nutrients and oxygen while removing waste---is the biological equivalent of power and network infrastructure.

\subsection{The Triad in Action: A Conceptual Example}

Consider assembling a simple bio-robotic system: a muscle that contracts in response to detected force.

\textbf{Components}:
\begin{itemize}
\item Muscle actuator (Actuator type)
\item Force sensor (Sensor type)
\item Neural controller (Processor type)
\end{itemize}

\textbf{Interfaces}:
\begin{itemize}
\item Sensor $\rightarrow$ Controller: electrical signal interface
\item Controller $\rightarrow$ Muscle: electrical stimulation interface
\item All components: perfusion interface for nutrients
\end{itemize}

\textbf{Runtime}:
\begin{itemize}
\item Perfusion system maintaining 37°C, pH 7.4
\item Monitoring system tracking viability and performance
\item Control loop executing feedback algorithm
\end{itemize}

In Bio-DSL (detailed in \S5):
\begin{lstlisting}
CONNECT sensor.output TO controller.input
CONNECT controller.output TO muscle.stimulation
RUNTIME { perfusion: standard_mammalian, control: closed_loop }
\end{lstlisting}

The power of this abstraction is that the same description could work with different muscle sources (human, mouse, synthetic), different sensor technologies (piezoelectric, biological), and different controller implementations (organoid, silicon chip). As long as interfaces are honored, components are interchangeable.


\section{Systematic Mapping: Software Engineering to Biological Engineering}

\subsection{Mapping Framework}

Not all software engineering concepts transfer equally to biology. We propose a three-category framework for analyzing mappings:

\begin{table}[h]
\centering
\caption{Mapping Categories}
\begin{tabular}{@{}lll@{}}
\toprule
Mapping Type & Definition & Transfer Difficulty \\
\midrule
Direct & Concept transfers with minimal adaptation & Low \\
Analogous & Core idea transfers but requires domain-specific adaptation & Medium \\
Novel & No software equivalent; requires new solutions & High \\
\bottomrule
\end{tabular}
\label{tab:mapping-categories}
\end{table}

\subsection{Direct Mappings}

These concepts can be transferred almost verbatim from software engineering:

\textbf{Semantic Versioning}: Software's Semantic Versioning (SemVer) specification defines version numbers as MAJOR.MINOR.PATCH. This transfers directly to Bio-Components:
\begin{itemize}
\item \textbf{MAJOR}: Interface-incompatible changes (e.g., different input signal type)
\item \textbf{MINOR}: Backward-compatible enhancements (e.g., improved force output)
\item \textbf{PATCH}: Optimizations without interface changes (e.g., faster response time)
\end{itemize}

Example: \texttt{muscle-actuator-human-skeletal@2.3.1}

\textbf{Dependency Declaration}: Software package manifests (package.json, requirements.txt) declare dependencies with version constraints. Bio-Component manifests can use identical syntax:
\begin{lstlisting}
dependencies:
  perfusion-medium: "DMEM@^1.0"
  oxygen-supply: ">=15%"
  temperature-control: "37+/-2C"
  co-culture:
    - "endothelial-cells@^1.2"
\end{lstlisting}

\textbf{Documentation Standards}: README files, API documentation, and usage examples transfer directly. A Bio-Component should include description, requirements, interface specification, usage examples, known limitations, and changelog.

\textbf{Licensing}: Software licenses (MIT, Apache, GPL) define usage rights. Biological components need similar frameworks covering usage rights, modification rights, sharing requirements, attribution, and commercial use restrictions.

\subsection{Analogous Mappings}

These concepts require adaptation but preserve core principles:

\textbf{Testing $\rightarrow$ Validation}:

\begin{table}[h]
\centering
\caption{Testing to Validation Mapping}
\begin{tabular}{@{}lll@{}}
\toprule
Software Testing & Biological Validation & Adaptation Notes \\
\midrule
Unit Test & Viability Test & Test single component function \\
Integration Test & Compatibility Test & Test component interactions \\
Stress Test & Endurance Test & Long-term, extreme conditions \\
Regression Test & Batch Consistency Test & New batches match previous \\
Performance Test & Efficiency Test & Output per resource consumed \\
\bottomrule
\end{tabular}
\label{tab:testing}
\end{table}

Key differences: Software tests are deterministic; biological tests are statistical. Software tests run in milliseconds; biological tests take days/weeks. Software tests are automated; biological tests require manual intervention.

Adaptation: Define acceptance criteria as statistical thresholds:
\begin{lstlisting}
tests:
  viability:
    metric: "cell_survival_rate"
    threshold: ">= 90%"
    confidence: "95%"
    sample_size: 10
\end{lstlisting}

\textbf{Error Handling $\rightarrow$ Failure Mode Management}:

Software distinguishes exceptions (catchable), errors (serious), and warnings (non-critical). Biological systems have analogous categories:
\begin{itemize}
\item \textbf{Recoverable Degradation}: Temporary performance drop (fatigue)
\item \textbf{Irreversible Damage}: Permanent function loss (necrosis)
\item \textbf{Systemic Risk}: Threats to other components (inflammation, infection)
\end{itemize}

\textbf{Logging $\rightarrow$ Biomarker Recording}:

Software logging captures timestamps, event types, contextual data, and stack traces. Biological logging captures timestamps, physiological measurements, environmental conditions, and anomaly indicators.

\subsection{Novel Challenges Requiring New Solutions}

These challenges have no direct software equivalent:

\textbf{Immune Compatibility}: Software components do not ``reject'' each other. Biological components from different sources may trigger immune responses.

Required Innovation---Immune Compatibility Protocol:
\begin{lstlisting}
immune_profile:
  mhc_class_i: ["HLA-A*02:01", "HLA-B*07:02"]
  mhc_class_ii: ["HLA-DR*04:01"]
  
compatibility_requirements:
  autologous: "preferred"
  allogeneic: "requires_matching"
  xenogeneic: "requires_isolation_barrier"
\end{lstlisting}

\textbf{Signal Crosstalk}: Software processes are isolated by operating system memory protection. Biological components share chemical environments where signals can interfere.

\textbf{Metabolic Coupling}: Software components consume CPU and memory independently. Biological components share metabolic resources and produce waste that affects neighbors.

Required Innovation---Metabolic Dependency Graph:
\begin{lstlisting}
metabolism:
  consumes:
    - glucose: "2.5 umol/hour"
    - oxygen: "5.0 umol/hour"
  produces:
    - lactate: "4.0 umol/hour"
    - co2: "4.5 umol/hour"
  toxic_threshold:
    lactate: "< 20 mM in shared medium"
\end{lstlisting}

\textbf{Living Degradation}: Software does not age. Biological components inherently degrade over time.

\textbf{Ethical Constraints}: Software has no inherent ethical status. Biological components, especially those derived from humans or involving neural tissue, raise ethical considerations with no software parallel.


\section{Bio-Component Specification: Design Rationale}

\subsection{Design Principles from Software Engineering}

The Bio-Component Specification draws from established software engineering principles:

\textbf{SOLID Principles Applied}:

\begin{table}[h]
\centering
\caption{SOLID Principles in Bio-Component Design}
\begin{tabular}{@{}p{3cm}p{4cm}p{5cm}@{}}
\toprule
Principle & Software Definition & Bio-Component Application \\
\midrule
Single Responsibility & A class should have one reason to change & A component should perform one biological function \\
Open/Closed & Open for extension, closed for modification & Components can be enhanced without changing interfaces \\
Liskov Substitution & Subtypes must be substitutable for base types & Compatible components must be interchangeable \\
Interface Segregation & Many specific interfaces over one general & Fine-grained interface definitions \\
Dependency Inversion & Depend on abstractions, not concretions & Depend on interface specs, not specific implementations \\
\bottomrule
\end{tabular}
\label{tab:solid}
\end{table}

\textbf{Convention over Configuration}: Borrowed from Ruby on Rails---provide sensible defaults to minimize required configuration.

\textbf{Schema-First Design}: Like OpenAPI/Swagger for REST APIs, we define the schema before implementations.

\subsection{Specification Structure}

The complete Bio-Component Spec schema includes:

\begin{lstlisting}
bio-component: "1.0"

# === IDENTIFICATION ===
info:
  id: string           # Unique identifier
  name: string         # Human-readable name
  version: string      # Semantic version
  description: string  # Brief description
  license: string      # Usage license
  authors: [string]    # Contributors

# === CLASSIFICATION ===
classification:
  type: enum [actuator, sensor, processor, metabolic, structural, connector]
  domain: string       # e.g., "musculoskeletal", "neural"
  tags: [string]       # Searchable tags

# === BIOLOGICAL SOURCE ===
source:
  organism: string     # e.g., "Homo sapiens"
  tissue_type: string  # e.g., "skeletal muscle"
  cell_types: [string] # e.g., ["myocyte", "fibroblast"]
  biosafety_level: enum [BSL-1, BSL-2, BSL-3]

# === INTERFACE DEFINITION ===
interface:
  inputs: [InputPort]
  outputs: [OutputPort]

# === ENVIRONMENTAL REQUIREMENTS ===
requirements:
  physical: PhysicalRequirements
  chemical: ChemicalRequirements
  biological: BiologicalRequirements

# === PERFORMANCE CHARACTERISTICS ===
performance:
  functional: FunctionalMetrics
  reliability: ReliabilityMetrics
  resources: ResourceConsumption

# === FAILURE MODES ===
failure_modes: [FailureMode]

# === TESTING ===
testing:
  unit_tests: [TestCase]
  integration_tests: [IntegrationTest]

# === DEPENDENCIES ===
dependencies:
  components: [ComponentDependency]
  adapters: [AdapterDependency]
\end{lstlisting}

\subsection{Versioning Strategy}

We adopt SemVer with biological interpretations:

\textbf{MAJOR version} (X.0.0): Interface-breaking changes
\begin{itemize}
\item Input/output port type changes
\item Required parameter additions
\item Environmental requirement changes that affect compatibility
\end{itemize}

\textbf{MINOR version} (0.X.0): Backward-compatible additions
\begin{itemize}
\item New optional output ports
\item Performance improvements
\item Additional monitoring capabilities
\end{itemize}

\textbf{PATCH version} (0.0.X): Backward-compatible fixes
\begin{itemize}
\item Protocol optimizations
\item Documentation updates
\item Minor performance tuning
\end{itemize}

Extended version format for biological specificity:
\begin{verbatim}
{version}+{batch}.{donor}.{modification}
Example: 2.3.1+batch20251228.donor42.wildtype
\end{verbatim}

\subsection{Example: Muscle Actuator Specification}

\begin{lstlisting}
bio-component: "1.0"

info:
  id: "muscle-actuator-human-skeletal"
  name: "Human Skeletal Muscle Actuator"
  version: "2.3.1"
  description: "Contractile muscle tissue for force generation"
  license: "CC-BY-SA-4.0"

classification:
  type: "actuator"
  domain: "musculoskeletal"
  tags: ["muscle", "contractile", "force-generation"]

source:
  organism: "Homo sapiens"
  tissue_type: "skeletal muscle"
  cell_types: ["myocyte", "satellite cell"]
  biosafety_level: "BSL-1"

interface:
  inputs:
    - id: "electrical_stimulation"
      type: "electrical"
      parameters:
        voltage: { range: [0, 5], unit: "V" }
        frequency: { range: [1, 100], unit: "Hz" }
  outputs:
    - id: "force_output"
      type: "mechanical"
      parameters:
        force: { range: [0, 50], unit: "mN" }

requirements:
  physical:
    temperature: { optimal: 37, range: [35, 39], unit: "C" }
  chemical:
    pH: { optimal: 7.4, range: [7.2, 7.6] }
    oxygen: { range: [15, 25], unit: "%" }

performance:
  functional:
    max_force: { value: 50, unit: "mN" }
    response_time: { typical: 150, max: 300, unit: "ms" }
  reliability:
    lifetime: { mean: 14, std: 3, unit: "days" }

failure_modes:
  - id: "fatigue"
    type: "recoverable"
    detection: "force_output < 80% baseline"
  - id: "necrosis"
    type: "irreversible"
    detection: "viability < 50%"
\end{lstlisting}


\section{Bio-DSL: Language Design Rationale}

\subsection{Why a Domain-Specific Language?}

Martin Fowler defines a domain-specific language (DSL) as ``a computer language specialized to a particular application domain.'' DSLs trade generality for expressiveness within their domain.

\begin{table}[h]
\centering
\caption{Benefits of DSLs for Biological Systems}
\begin{tabular}{@{}lll@{}}
\toprule
Benefit & Explanation & Bio-DSL Application \\
\midrule
Expressiveness & Say more with less & Describe complex assemblies concisely \\
Readability & Domain experts can understand & Biologists can read system descriptions \\
Validation & Domain-specific error checking & Catch interface mismatches at ``compile time'' \\
Abstraction & Hide implementation details & Focus on what, not how \\
\bottomrule
\end{tabular}
\label{tab:dsl-benefits}
\end{table}

\subsection{Design Goals}

\begin{enumerate}
\item \textbf{Declarative}: Describe \textit{what} the system is, not \textit{how} to build it
\item \textbf{Readable}: A biologist should understand the intent without programming background
\item \textbf{Verifiable}: Static analysis can catch errors before physical assembly
\item \textbf{Executable}: Can generate runtime configurations and monitoring dashboards
\item \textbf{Composable}: Systems can be nested and reused
\end{enumerate}

\subsection{Language Constructs}

\textbf{Component Declaration}:
\begin{lstlisting}
// Import component from registry with version constraint
COMPONENT <alias> FROM "<source>@<version>" [AS <local_name>]

// Examples:
COMPONENT flexor FROM "muscle-actuator-human-skeletal@^2.3"
COMPONENT sensor FROM "piezo-force-sensor@~1.1" AS force_sensor
COMPONENT controller FROM "neural-organoid-spinal@>=0.8"
\end{lstlisting}

\textbf{Connection Declaration}:
\begin{lstlisting}
// Basic connection
CONNECT <source>.<port> TO <target>.<port>

// Connection with adapter
CONNECT <source>.<port> TO <target>.<port> VIA <adapter>

// Examples:
CONNECT sensor.output TO controller.input
CONNECT controller.output TO flexor.stimulation VIA signal_converter
\end{lstlisting}

\textbf{Runtime Configuration}:
\begin{lstlisting}
RUNTIME {
  perfusion: {
    medium: "DMEM + 10% FBS",
    temperature: 37 C,
    pH: 7.4,
    flow_rate: 0.5 mL/min
  },
  control: {
    mode: "closed_loop",
    algorithm: "PID",
    parameters: { Kp: 0.8, Ki: 0.2, Kd: 0.1 }
  },
  monitoring: {
    log_interval: 10 s,
    metrics: ["force", "viability", "temperature"],
    alerts: {
      "viability < 80%": "WARNING",
      "temperature > 39C": "CRITICAL"
    }
  }
}
\end{lstlisting}

\textbf{Behavioral Logic}:
\begin{lstlisting}
ON STARTUP DO {
  SET perfusion.flow_rate = 0.5 mL/min
  WAIT 300 s  // Equilibration
  SET controller.mode = "active"
}

WHEN flexor.fatigue_index > 0.3 THEN {
  LOG "Fatigue detected"
  REDUCE flexor.stimulation_frequency BY 20%
}

EVERY 1 hour DO {
  RUN viability_check()
  IF any.viability < 85% THEN {
    INCREASE perfusion.flow_rate BY 10%
  }
}
\end{lstlisting}

\textbf{Test Declaration}:
\begin{lstlisting}
TEST contraction_response {
  description: "Verify muscle responds to stimulation"
  
  GIVEN flexor.state == "ready"
  WHEN STIMULATE flexor AT 10 Hz, 2 V FOR 1 s
  THEN EXPECT flexor.force IN [5, 15] mN WITHIN 200 ms
}
\end{lstlisting}

\subsection{Comparison with Related Languages}

\begin{table}[h]
\centering
\caption{Comparison with Existing Biological Languages}
\begin{tabular}{@{}llll@{}}
\toprule
Language & Abstraction Level & Purpose & Relationship to Bio-DSL \\
\midrule
SBOL & Genetic & DNA sequence description & Lower level; component internals \\
SBML & Molecular & Biochemical reaction networks & Lower level; component dynamics \\
CellML & Cellular & Cell mathematical models & Lower level; behavior models \\
Bio-DSL & Organ/System & Component composition & Higher level; system assembly \\
\bottomrule
\end{tabular}
\label{tab:language-comparison}
\end{table}

Bio-DSL is designed to \textbf{complement} these languages: Use SBOL to describe genetic modifications within a component; use SBML to model biochemical behavior; use Bio-DSL to describe how components connect into systems.

\subsection{Tooling Vision}

A complete Bio-DSL ecosystem would include:
\begin{enumerate}
\item \textbf{Parser/Validator}: Check syntax and semantic correctness
\item \textbf{Type Checker}: Verify interface compatibility
\item \textbf{Simulator}: Predict system behavior before physical assembly
\item \textbf{Code Generator}: Produce runtime configurations
\item \textbf{Visual Editor}: Drag-and-drop system design
\item \textbf{Package Manager}: Discover and install components
\item \textbf{Test Runner}: Execute test suites
\item \textbf{Documentation Generator}: Produce human-readable specs
\end{enumerate}


\section{Fundamental Differences and Open Challenges}

While the paradigm transfer from software to biological engineering is powerful, fundamental differences between the domains create challenges that require novel solutions beyond direct mapping.

\subsection{Determinism vs. Stochasticity}

\begin{table}[h]
\centering
\caption{Determinism Comparison}
\begin{tabular}{@{}ll@{}}
\toprule
Software & Biology \\
\midrule
Function calls always return & Cells may die unexpectedly \\
Same input $\rightarrow$ same output & Biological variability is inherent \\
Errors can be precisely located & Failure modes are complex and interacting \\
State is fully observable & Internal state is partially hidden \\
\bottomrule
\end{tabular}
\label{tab:determinism}
\end{table}

A software function \texttt{add(2, 3)} will always return \texttt{5}. A biological muscle stimulated with identical parameters will produce slightly different force each time, and occasionally may not respond at all.

\textbf{Implications}: Interface contracts must be probabilistic; testing must be statistical; runtime must handle uncertainty through redundant components and graceful degradation strategies.

\subsection{Discrete vs. Continuous}

Software state transitions are instantaneous: a variable is either \texttt{true} or \texttt{false}. Biological state transitions are gradual: a muscle doesn't switch from ``relaxed'' to ``contracted'' but transitions through a continuum.

\textbf{Implications}: Interface parameters need tolerance ranges; state definitions need thresholds; timing specifications need ranges rather than exact values.

\subsection{Isolation vs. Coupling}

Software processes are isolated by the operating system. A bug in one process cannot corrupt another's memory. Biological components share culture medium, and one component's metabolic waste affects all others.

\textbf{Implications}: Explicit coupling declarations; isolation adapter specifications; system-level resource budgeting.

\subsection{The Immune System: No Software Equivalent}

Software components do not ``reject'' each other. Biological components from different genetic backgrounds may trigger immune responses ranging from mild inflammation to complete destruction.

\textbf{Required Innovations}:
\begin{itemize}
\item Immune compatibility scoring
\item Isolation barrier specifications
\item Compatibility checking in Bio-DSL compiler
\end{itemize}

\subsection{Living Degradation}

Software does not age. A function written in 1990 executes identically today (given compatible runtime). Biological components inherently degrade: cells senesce, proteins denature, structures weaken.

\textbf{Required Innovations}:
\begin{itemize}
\item Degradation modeling (e.g., Weibull distribution)
\item Maintenance protocols
\item Replacement strategies (hot-swap with backup)
\end{itemize}

\subsection{Ethical Constraints}

Software has no inherent ethical status. Biological components, especially those involving human cells or neural tissue, raise ethical considerations:
\begin{itemize}
\item \textbf{Source ethics}: How were cells obtained? Was there informed consent?
\item \textbf{Capability ethics}: Could the assembly develop consciousness?
\item \textbf{Use ethics}: What applications are acceptable?
\item \textbf{Disposal ethics}: How should biological materials be destroyed?
\end{itemize}

\textbf{Required Innovations}: Ethical metadata in specifications; capability limits enforced by Bio-DSL compiler.

\subsection{Summary: The Innovation Agenda}

\begin{table}[h]
\centering
\caption{Innovation Agenda Summary}
\begin{tabular}{@{}lll@{}}
\toprule
Challenge & Software Equivalent & Required Innovation \\
\midrule
Stochasticity & None (deterministic) & Probabilistic contracts, statistical testing \\
Continuous states & None (discrete) & Tolerance ranges, threshold definitions \\
Metabolic coupling & None (isolated) & Coupling declarations, resource budgeting \\
Immune rejection & None & Compatibility scoring, isolation barriers \\
Living degradation & None & Degradation models, maintenance protocols \\
Ethical constraints & Licensing (weak) & Ethical metadata, capability limits \\
\bottomrule
\end{tabular}
\label{tab:innovation-agenda}
\end{table}

These challenges do not invalidate the paradigm transfer---they define the research agenda for making it complete.


\section{Related Work and Positioning}

\subsection{Synthetic Biology and Standardization}

The BioBricks Foundation and iGEM (International Genetically Engineered Machine) competition pioneered biological standardization at the genetic level. BioBricks defined standard assembly methods for DNA parts.

SBOL (Synthetic Biology Open Language) provides a standardized data format for describing genetic designs, enabling exchange between software tools and laboratories.

\textbf{Relationship to Wetware Engineering}:
\begin{itemize}
\item BioBricks/SBOL operate at the \textbf{genetic level} (DNA sequences)
\item Wetware Engineering operates at the \textbf{organ/system level} (tissues, organoids)
\item They are \textbf{complementary}: BioBricks could define the genetic content \textit{within} a Bio-Component
\end{itemize}

\subsection{Organ-on-Chip and Organoids}

Organ-on-chip devices culture human cells in microfluidic environments that mimic organ physiology. Organoids are self-organizing 3D tissue cultures that recapitulate organ structure and function.

\textbf{Relationship to Wetware Engineering}:
\begin{itemize}
\item Organ-on-chip provides \textbf{physical implementations} of Bio-Components
\item Current systems lack \textbf{standardized interfaces} between chips
\item Wetware Engineering provides the \textbf{abstraction framework} they need
\end{itemize}

\subsection{Systems Biology Modeling}

SBML (Systems Biology Markup Language) represents computational models of biological processes. CellML describes mathematical models of cellular function.

\textbf{Relationship to Wetware Engineering}:
\begin{itemize}
\item SBML/CellML describe \textbf{how components behave internally} (simulation)
\item Bio-DSL describes \textbf{how components connect externally} (composition)
\item They serve different purposes and can be used together
\end{itemize}

\subsection{Biohybrid Robotics}

Research groups have created robots powered by biological actuators: muscle-powered swimmers, insect-machine hybrids, and biohybrid grippers. These works prove biological components can be engineered.

\textbf{Relationship to Wetware Engineering}: Current biohybrid work is bespoke---each system designed from scratch. Wetware Engineering asks: how do we make this \textbf{systematic and reproducible}?

\subsection{Positioning Summary}

Wetware Engineering's unique contribution: Providing the \textbf{system-level abstraction layer} that connects molecular/genetic engineering to functional biological systems, using software engineering principles.

\subsection{What We Are NOT Claiming}

To be clear about scope:
\begin{enumerate}
\item \textbf{We are not claiming to have built working systems}. This paper proposes a framework; implementation is future work.
\item \textbf{We are not claiming biology is ``just like'' software}. Section 6 details fundamental differences requiring novel solutions.
\item \textbf{We are not claiming to replace existing approaches}. We complement synthetic biology, organoid research, and systems biology.
\item \textbf{We are not claiming immediate practical application}. The roadmap spans decades.
\end{enumerate}

What we ARE claiming: \textbf{The conceptual framework of software engineering---modularity, interfaces, composition, versioning---provides valuable abstractions for biological system construction, and articulating this framework is a necessary first step.}


\section{Conclusion and Future Directions}

\subsection{Summary of Contributions}

This paper has proposed \textbf{Wetware Engineering}, a cross-disciplinary methodology that systematically transfers software engineering paradigms to biological system construction. Our contributions are:

\textbf{Conceptual Framework}: We defined the \textbf{Component-Interface-Runtime triad} as the foundational abstraction for modular biological systems.

\textbf{Technical Specifications}: We proposed concrete specifications:
\begin{itemize}
\item \textbf{Bio-Component Spec v0.1}: A standardized schema for describing biological modules
\item \textbf{Bio-DSL}: A domain-specific language for declarative system composition
\end{itemize}

\textbf{Systematic Analysis}: We provided systematic mappings between software and biological engineering, categorized as Direct, Analogous, or Novel.

\textbf{Honest Assessment}: We identified fundamental differences that require innovation beyond paradigm transfer, establishing a research agenda for the field.

\subsection{The Path Forward}

\textbf{Phase 1: Foundation (1-3 years)}
\begin{itemize}
\item Refine specifications based on community feedback
\item Develop proof-of-concept tooling (parser, validator)
\item Document 10-20 existing biological systems using Bio-Component Spec
\item Publish reference implementations
\end{itemize}

\textbf{Phase 2: Validation (3-7 years)}
\begin{itemize}
\item Physically implement 2-3 component systems using the framework
\item Validate that standardized descriptions improve reproducibility
\item Develop interface adapters for common connection types
\item Build component registry infrastructure
\end{itemize}

\textbf{Phase 3: Ecosystem (7-15 years)}
\begin{itemize}
\item Establish community governance for standards
\item Commercial adoption in drug discovery, tissue engineering
\item Educational curriculum development
\item International standardization (ISO, IEEE)
\end{itemize}

\subsection{Call to Action}

Wetware Engineering requires contributions from multiple communities:

\textbf{For Biologists and Tissue Engineers}: Describe your work using Bio-Component Spec format; identify interface requirements; share protocols in machine-readable formats.

\textbf{For Software Engineers}: Contribute tooling (parsers, validators, editors); apply design patterns to biological contexts; develop testing frameworks adapted for biological variability.

\textbf{For Standards Bodies}: Engage early in specification development; coordinate with existing biological standards (SBOL, SBML).

\textbf{For Funding Agencies}: Support cross-disciplinary methodology research; fund infrastructure (registries, tools) not just applications.

\subsection{Limitations and Caveats}

We acknowledge significant limitations:
\begin{enumerate}
\item \textbf{No experimental validation}: This paper proposes a framework; we have not physically built systems using it.
\item \textbf{Specification incompleteness}: Bio-Component Spec v0.1 is a starting point, not a finished standard.
\item \textbf{Tooling absence}: The envisioned toolchain does not yet exist.
\item \textbf{Community adoption uncertainty}: Standards succeed through adoption, which cannot be guaranteed.
\item \textbf{Biological complexity}: Real biological systems may resist the clean abstractions we propose.
\end{enumerate}

These limitations do not invalidate the approach---they define the work ahead.

\subsection{Closing Thoughts}

Software engineering transformed from craft to discipline over five decades. The journey included conceptual breakthroughs, standardization efforts, tool development, community building, and educational formalization.

Biological engineering stands at a similar inflection point. The question is not whether modularization will come to biology---the complexity of biological systems demands it---but how quickly and how well.

We offer Wetware Engineering not as a finished solution, but as a \textbf{conceptual framework} and \textbf{conversation starter}. The goal is to accelerate the transition from artisanal biological construction to systematic biological engineering.

\begin{quote}
``Software engineering took 50 years to evolve from monolithic applications to microservices architecture. We hope biological engineering won't need another 50 years.''
\end{quote}

The tools of software engineering---abstraction, modularity, standardization, composition---are not specific to silicon. They are \textbf{general principles of managing complexity}. Biology is complex. These principles can help.

The future of biological engineering is modular. The question is: will we design that future deliberately, or stumble into it accidentally?

We choose to design.

\section*{Acknowledgments}

The author thanks the open-source software engineering community for decades of accumulated wisdom that made this cross-disciplinary transfer possible.

\section*{Data Availability}

All specifications and examples are available at: \url{https://github.com/tukuaiai/wetware-engineering}


\bibliographystyle{unsrtnat}

\begin{thebibliography}{34}

\bibitem{parnas1972}
Parnas, D. L. (1972). On the criteria to be used in decomposing systems into modules. \textit{Communications of the ACM}, 15(12), 1053-1058.

\bibitem{dijkstra1968}
Dijkstra, E. W. (1968). Go to statement considered harmful. \textit{Communications of the ACM}, 11(3), 147-148.

\bibitem{gamma1994}
Gamma, E., Helm, R., Johnson, R., \& Vlissides, J. (1994). \textit{Design Patterns: Elements of Reusable Object-Oriented Software}. Addison-Wesley.

\bibitem{martin2003}
Martin, R. C. (2003). \textit{Agile Software Development: Principles, Patterns, and Practices}. Prentice Hall.

\bibitem{fowler2010}
Fowler, M. (2010). \textit{Domain-Specific Languages}. Addison-Wesley.

\bibitem{brooks1975}
Brooks, F. P. (1975). \textit{The Mythical Man-Month: Essays on Software Engineering}. Addison-Wesley.

\bibitem{newman2015}
Newman, S. (2015). \textit{Building Microservices: Designing Fine-Grained Systems}. O'Reilly Media.

\bibitem{semver}
Preston-Werner, T. (2013). Semantic Versioning 2.0.0. https://semver.org/

\bibitem{endy2005}
Endy, D. (2005). Foundations for engineering biology. \textit{Nature}, 438(7067), 449-453.

\bibitem{canton2008}
Canton, B., Labno, A., \& Endy, D. (2008). Refinement and standardization of synthetic biological parts and devices. \textit{Nature Biotechnology}, 26(7), 787-793.

\bibitem{galdzicki2014}
Galdzicki, M., et al. (2014). The Synthetic Biology Open Language (SBOL) provides a community standard for communicating designs in synthetic biology. \textit{Nature Biotechnology}, 32(6), 545-550.

\bibitem{beal2020}
Beal, J., et al. (2020). Communicating structure and function in synthetic biology diagrams. \textit{ACS Synthetic Biology}, 9(8), 2025-2040.

\bibitem{lancaster2014}
Lancaster, M. A., \& Knoblich, J. A. (2014). Organogenesis in a dish: Modeling development and disease using organoid technologies. \textit{Science}, 345(6194), 1247125.

\bibitem{takebe2019}
Takebe, T., \& Wells, J. M. (2019). Organoids by design. \textit{Science}, 364(6444), 956-959.

\bibitem{clevers2016}
Clevers, H. (2016). Modeling development and disease with organoids. \textit{Cell}, 165(7), 1586-1597.

\bibitem{rossi2018}
Rossi, G., Manfrin, A., \& Lutolf, M. P. (2018). Progress and potential in organoid research. \textit{Nature Reviews Genetics}, 19(11), 671-687.

\bibitem{bhatia2014}
Bhatia, S. N., \& Ingber, D. E. (2014). Microfluidic organs-on-chips. \textit{Nature Biotechnology}, 32(8), 760-772.

\bibitem{huh2010}
Huh, D., et al. (2010). Reconstituting organ-level lung functions on a chip. \textit{Science}, 328(5986), 1662-1668.

\bibitem{ronaldson2018}
Ronaldson-Bouchard, K., \& Vunjak-Novakovic, G. (2018). Organs-on-a-chip: A fast track for engineered human tissues in drug development. \textit{Cell Stem Cell}, 22(3), 310-324.

\bibitem{hucka2003}
Hucka, M., et al. (2003). The systems biology markup language (SBML): A medium for representation and exchange of biochemical network models. \textit{Bioinformatics}, 19(4), 524-531.

\bibitem{lloyd2004}
Lloyd, C. M., Halstead, M. D., \& Nielsen, P. F. (2004). CellML: Its future, present and past. \textit{Progress in Biophysics and Molecular Biology}, 85(2-3), 433-450.

\bibitem{raman2017}
Raman, R., \& Bashir, R. (2017). Biomimicry, biofabrication, and biohybrid systems: The emergence and evolution of biological design. \textit{Advanced Healthcare Materials}, 6(20), 1700496.

\bibitem{ricotti2017}
Ricotti, L., et al. (2017). Biohybrid actuators for robotics: A review of devices actuated by living cells. \textit{Science Robotics}, 2(12), eaaq0495.

\bibitem{park2016}
Park, S. J., et al. (2016). Phototactic guidance of a tissue-engineered soft-robotic ray. \textit{Science}, 353(6295), 158-162.

\bibitem{cvetkovic2014}
Cvetkovic, C., et al. (2014). Three-dimensionally printed biological machines powered by skeletal muscle. \textit{Proceedings of the National Academy of Sciences}, 111(28), 10125-10130.

\bibitem{musk2019}
Musk, E., \& Neuralink. (2019). An integrated brain-machine interface platform with thousands of channels. \textit{Journal of Medical Internet Research}, 21(10), e16194.

\bibitem{lebedev2017}
Lebedev, M. A., \& Nicolelis, M. A. (2017). Brain-machine interfaces: From basic science to neuroprostheses and neurorehabilitation. \textit{Physiological Reviews}, 97(2), 767-837.

\bibitem{yuste2017}
Yuste, R., et al. (2017). Four ethical priorities for neurotechnologies and AI. \textit{Nature}, 551(7679), 159-163.

\bibitem{ienca2017}
Ienca, M., \& Andorno, R. (2017). Towards new human rights in the age of neuroscience and neurotechnology. \textit{Life Sciences, Society and Policy}, 13(1), 5.

\bibitem{hyun2020}
Hyun, I., Scharf-Deering, J. C., \& Lunshof, J. E. (2020). Ethical issues related to brain organoid research. \textit{Brain Research}, 1732, 146653.

\bibitem{simon1996}
Simon, H. A. (1996). \textit{The Sciences of the Artificial} (3rd ed.). MIT Press.

\bibitem{alexander1977}
Alexander, C., Ishikawa, S., \& Silverstein, M. (1977). \textit{A Pattern Language: Towns, Buildings, Construction}. Oxford University Press.

\bibitem{kuhn1962}
Kuhn, T. S. (1962). \textit{The Structure of Scientific Revolutions}. University of Chicago Press.

\end{thebibliography}

\end{document}
